\documentclass[10pt,letterpaper]{article}
\usepackage[latin1]{inputenc}
\usepackage{amsmath}
\usepackage{amsfonts}
\usepackage{amssymb}
\usepackage{graphicx}
\usepackage{hyperref}
\usepackage{listings}


\author{Andrew Rosen}
\title{Reasonably Priced Pizza}
\date{}

\begin{document}
	
	\maketitle
	\section{Premise}
	Everyone likes pizza.\footnote{Those who don't are heretics.}
	Everyone also likes making sure that their pizza is the best deal.\footnote{Especially undergraduate and graduate students who tend to be a bit strapped for cash.}
	
	Today's task is to write a program to make sure you get the best deal for your pizza.
	How are we going to define best deal? 
	Since making a huge database of all the pizza shops in Philadelphia and aggregating the reviews and the prices seems like a whole lot of work, not to mention subjective, we are going to use \textit{price-per-square inch}.\footnote{This will really only work for regular or New York style pizzas, as opposed to Chicago style pizzas, which have a depth to them.}

	\section{Program Specifications} 
	
	Because we are just starting to learn programming, our program will be pretty simple.
	It will ask the user to input the size of the pizza and ask the user the price.
	It will then use this information to compute price per square inch.
	
	We are going to break this assignment into multiple methods, even though it is fairly simple.
	Why?
	\begin{itemize}
		\item This is good practice for writing methods and understanding how they work.
		\item Writing methods in this manner that do a clearly defined task allows you to reuse them in other programs.  We call this \textit{modularity}.
		\item The most important reason is that this allows you to write small parts of your program and ensure they are working.  
		Too often we see students try to write the whole program at once and then try to fix all the errors.  
		These students are stressed and never score better than a C.
		
		Instead, write just the method that calculates the area of the pizza (see below) , test it in the \texttt{main} method, confirm it works, and then move on to the next part and repeat.
	\end{itemize}
	
	\subsection{Area of a Pizza Pie}
	Pizzas in the USA are just big circles of bread, cheese, and tomato sauce, measured by the size of their diameter in inches.
	You can use that to get the area of the circle, which is measured in square inches.
	
	\textbf{Write a method} which takes in the diameter of a circle and returns the area.
	Remember:
	
	\[d =  2r\]
	\[a =  2\pi r\]
	
	
	
	
	\subsection{Computing the Price per Square Inch}
	Figuring out the price per square inch should be it's own method.
	Given the area of a circle in square inches and the price, the method should return the price per square inch.
	
	\subsection{Interactivity}
	Your program should be interactive.  
	This means that the program should ask the user to enter the size of the pizza and the price.\footnote{Your program does not have to interact with the user in a professional manner, but you must spell correctly. }
	
	All the input from console and output to  the user\footnote{Meaning printing.} should happen on in \texttt{main}.
	
	
	
	
	\section{Grading Criteria}
	
	\begin{description}
		\item[30 points] The program calculates the price per square inch of a pizza.  
		\item[30 points] The program calls methods.
		\item[10 points] Program is interactive.
		\item[10 points]  In addition, all the interactivity is inside of the main method.  There are no \texttt{println} statements or reading from a Scanner except in main.
		
		\item[15 points] The source code is reasonably formatted.
		\item[5 points] Correct spelling is used in the outputs
		
		\item[5 points EC] Program has \texttt{while} loops to enter more pizzas and figures out the cheapest pizza of all entered options.  This requires reading ahead.
		\item[5 points EC] Add in support for Chicago style pizzas.  This is left to the discretion of the students.
	\end{description}
\end{document}
