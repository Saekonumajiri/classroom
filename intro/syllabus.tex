\documentclass[10pt, letter]{article}
\usepackage{fancyhdr, lastpage}
\usepackage[english]{babel}
\usepackage[super]{nth}

\usepackage{amsmath}
\usepackage{graphicx}
%\usepackage[colorinlistoftodos]{todonotes}
\usepackage{hyperref}

\pagestyle{fancy}
\setlength{\headheight}{15.2pt}
\rhead{Syllabus}
\lhead{Program Design and Abstraction}

\title{Syllabus \\ CIS 1068 - Program Design and Abstraction  \\ Fall 2017}
\author{Dr. Andrew Rosen  \\ andrew.rosen@temple.edu}
\date{}
\begin{document}
\maketitle

\section{Course Details}

An introduction to discipline of programming.  We will cover of introductory programming in Java, and well as an introduction to Object-Oriented programming via inheritance, polymorphism, and encapsulation.  The final weeks will cover exception handling and file i/o.
Students will be expected to be able to write a program that will autonomously crawl through a website for resources of interest and capture them.
This is a 4 credit-hour course.
%There is an assumption of \textbf{some programming knowledge} and math skills.




\subsection*{Time and Place}
Monday and Wednesday 4:00pm-5:20pm SERC 110B
\\
Labs are on Friday at 11:00am and 1:00pm in SERC 206.
\\
Please attend the lab on your schedule.


\subsection*{Prerequisites}
C or better in Mathematics 1021 or higher, or placement into Mathematics 1022 or higher. 
Students are expected to know how to use a computer.
\textbf{We strongly recommend \textit{all} students who have no programming experience to take CIS 1051 or 1057 first}.

%A laptop is strongly recommended, but not required.\footnote{If you cannot afford a computer, please see me as soon as possible to discuss solutions.}



\subsection*{Textbook}
\begin{itemize}
	\item Stuart Reges \& and Marty Stepp: Building Java Programs a Back To Basics Approach, 4th edition
\end{itemize}

I will be roughly following the material presented in the book.
Having the textbook is  strongly recommended and a useful aid for your initial education.
I also suggest reading the following if you need an additional reference.

\begin{itemize}
	\item Allen Downey \& Chris Mayfield: Think Java: How to Think Like a Computer Scientist
\end{itemize}

You can find it for free \href{http://greenteapress.com/wp/think-java/}{here}.   I suggest checking out the interactive version.

\subsection*{Office hours}
My office is SERC 349.
My office hours are Monday and Wednesday 2:30-3:30, and by appointment.
If you cannot meet me during my office hours or need to meet me for help at some other time, please email me to arrange an appointment. 
I can meet via Webex at odd hours.


\subsection*{TA Information}
Forthcoming

{\footnotesize
	\begin{tabular}{l l l l l}
		TA & Email &Office Hours & Office Location \\ \hline

		Alexey Uversky &tuc70064@temple.edu& Monday 12:00 - 2:30pm & SERC 349 \\
		
		 & &Tuesday 11:00 - 1:00pm & \\
	\end{tabular}
}





%\subsection*{Course Goals}
%This course will cover:

%\begin{itemize}
%	\item Learn what Computer Science is.
%	\item Java programming 
%	\item Basic Object-Oriented design.
%	\begin{itemize}
%		\item Inheritance
%		\item Encapsulation
%		\item Polymorphism
%	\end{itemize}
%	\item Learn how to think and solve problems.
%\end{itemize}



\section{Expectations}

\subsection*{Attendance}
Attendance is expected for both lectures and labs.  
If you arrive late, please do your best to enter quietly.  
Lectures will cover more content than is present in the book and it is highly unlikely you will succeed without attending lectures.
I am also prone to mentioning what will and won't be on the test, up to and including exact questions.



\subsection*{Academic Honesty}
All work submitted for grading must be the student's own work. A student that submits an assignment that copies the work of another, in whole or in part, will be assigned a grade of zero for that assignment. Any student found to be cheating on an examination will receive a score of zero for that exam. Cheating on an assignment or exam may result in dismissal from the course and notification of the Dean of Students.

When in doubt, provide a citation.
Working with others and creating your own unique solutions is not cheating.
We encourage you to ask your classmates for help and to collaborate.
Copying someone else's work is cheating and will be dealt with accordingly.


\subsection*{Disabilities}
Any student who has a need for accommodation based on the impact of a documented disability should contact me so we can privately discuss how I can help. 
If you have not done so already, please contact Disability Resources and Services (DRS) at 215-204-1280 in 100 Ritter Annex to learn more about the resources available to you.
I will work with DRS to coordinate reasonable accommodations for all students with documented disabilities.


\subsection*{Recordings}
I will be recording my lectures via my laptop.
This is not meant as a substitute for class, but as a study aid, as my webcam cannot capture the entire whiteboard.

\section{Assignments and Labs}
Assignments and Labs are intrinsically tied.
Your assignments are posted electronically on Blackboard.
You will receive an assignment approximately every week on Friday and due at 6am on Saturday the following week.

Lab sessions are dedicated time for you to work on assignments.
I or the TA will explain the assignment at the beginning of lab and we will be available to aid you and answer any questions you have.
Attendance is expected.


\subsection*{Compilation}
Programming assignments are expected to compile.  If your program does not compile and it is sent to us without informing us of a compilation issue, this indicates that you did not take the time to check your own work.  \textbf{You will receive a zero for the assignment if your program fails to compile and you did not warn us.}

If you have trouble fixing a compilation issue, do the following:
{\footnotesize
	\begin{itemize}
		\item Read the error and find the line number.
		\item Check that your parenthesis/braces/brackets match.
		\item Use Google.
		\item Email your TA or me.
	\end{itemize}
}
\subsection*{File Formats}
When submitting programming assignments, you must submit your source code and not the compiled program.
Your source code must be a \emph{.java} file, not a \emph{.class} file.
\textbf{If you submit a .txt, .doc, .rtf, or anything else that is not a .java file, you will receive a zero for that assignment}.




\subsection*{Turning in}
To get a grade on your assignment, it must be 
\begin{enumerate}
	\item Submitted online to Blackboard.  This is so we have an electronic ``paper trail'' for your assignment.
	\item Demoed to the me or a TA.  Demoing your work involves showing that your program works, explaining pieces of your source code, and answering some questions.
\end{enumerate}

We accept your late work. 



\subsection*{Late Policy}
Late assignments will be accepted with a penalty, as described below.  $$grade = score \cdot 0.95^{l}$$
Where $l$ is the number of days late.  While this looks a bit intimidating, this scheme does provide more points than just a straight deduction every day. \textbf{It is your responsibility to demo late work.  Work not demoed will receive a 0 at the end of the course.}

Late grades are determined by when you turn it in, not when you demo it, but the sooner you demo, the sooner you can correct a possible mistake. 

\subsubsection*{Example}
Alice turns in an assignment that would have gotten an 100, but it's 5 days late.  Her grade is $$100\cdot 0.95^{5} = 77.38$$ A 77 isn't ideal, but it's within striking range of a B and a great deal more than 0.

\section{Exams}
Exams make up the majority of your grade.
Practice exams may be provided and reviewed.  
Please do not miss any exam.  
Makeup exams will be given on a case by case basis.


You can expect two exams and a final, with the first exam occurring during the week 5 lab and the second occurring at the week 10 lab.
This is subject to change.


\subsection{{Final Exam}}
You will take a common final exam on December 19th at 3:30 to 5:30.  
If you have more than two finals on the same day, university policy allows you to request a professor provide a different time to take your final.


\section{Grading}

Letter grades will be assigned at no higher than A$\geq$90, B$\geq$80, etc.
In general, you need above a 70 to pass the class.
There may be extra credit opportunities available in on exams and on assignments.
I will not give out extra assignments or tests for you to make up your grade, but I usually drop a lab.
Plus/minus grading will be used.
\subsection*{Composition}
Your final grade in the class will be the average of each item, weighted accordingly.

\begin{tabular}{ l  r l}
	Participation and Attendance & 5 \% & at most \\
	Labs & 30 \% & at most \\
	Exams & 65 \% & at least \\
\end{tabular}

\section{Disclaimer}
The syllabus is here to serve as a guide and may be subject to changes.  Up to date information, assignments, and class material can be found on online.

This syllabus may be updated to reflect changes.


\subsection*{Students and Faculty Academic Rights and Responsibilities}
\href{http://policies.temple.edu/PDF/99.pdf}{Student and Faculty Academic Rights and Responsibilities} can be found in the link.


\section{Tips for Succeeding}
\begin{itemize}
	\item Don't be afraid to ask questions in class.  If you have a question, I guarantee another student has the same question.  
	\item This goes doubly so for math.  If you don't understand or remember a concept when we bring it up \emph{let us know.}
	\item Get into the habit of studying a couple of days early.
	\item Do your homework.
	\item Give yourself more time than you think you need.
	\item Use a clear, easy-to-read, monospaced font while coding.
	\item Do your homework.
	\item Ask questions.  Take advantage of our office hours.
	\item Do your homework.  In the previous classes I've taught, the vast majority of students who turned in all their homeworks managed to earn A's and B's, even when they had a bad day on a test.
	\item Likewise, not turning in homeworks corresponds very strongly with failing.
	\item Do your homework. Seriously.  It is nigh impossible to pass without doing your homework.
	
	
\end{itemize}





\end{document}